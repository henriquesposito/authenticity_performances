% Options for packages loaded elsewhere
\PassOptionsToPackage{unicode}{hyperref}
\PassOptionsToPackage{hyphens}{url}
%
\documentclass[
  12pt,
]{article}
\usepackage{amsmath,amssymb}
\usepackage{lmodern}
\usepackage{iftex}
\ifPDFTeX
  \usepackage[T1]{fontenc}
  \usepackage[utf8]{inputenc}
  \usepackage{textcomp} % provide euro and other symbols
\else % if luatex or xetex
  \usepackage{unicode-math}
  \defaultfontfeatures{Scale=MatchLowercase}
  \defaultfontfeatures[\rmfamily]{Ligatures=TeX,Scale=1}
  \setmainfont[]{Times New Roman}
\fi
% Use upquote if available, for straight quotes in verbatim environments
\IfFileExists{upquote.sty}{\usepackage{upquote}}{}
\IfFileExists{microtype.sty}{% use microtype if available
  \usepackage[]{microtype}
  \UseMicrotypeSet[protrusion]{basicmath} % disable protrusion for tt fonts
}{}
\makeatletter
\@ifundefined{KOMAClassName}{% if non-KOMA class
  \IfFileExists{parskip.sty}{%
    \usepackage{parskip}
  }{% else
    \setlength{\parindent}{0pt}
    \setlength{\parskip}{6pt plus 2pt minus 1pt}}
}{% if KOMA class
  \KOMAoptions{parskip=half}}
\makeatother
\usepackage{xcolor}
\IfFileExists{xurl.sty}{\usepackage{xurl}}{} % add URL line breaks if available
\IfFileExists{bookmark.sty}{\usepackage{bookmark}}{\usepackage{hyperref}}
\hypersetup{
  pdftitle={Performing Authenticity: Anti-Political Correctness as an Authenticity Performances in Politics in Brazil and the United States},
  hidelinks,
  pdfcreator={LaTeX via pandoc}}
\urlstyle{same} % disable monospaced font for URLs
\usepackage[margin=1in]{geometry}
\usepackage{graphicx}
\makeatletter
\def\maxwidth{\ifdim\Gin@nat@width>\linewidth\linewidth\else\Gin@nat@width\fi}
\def\maxheight{\ifdim\Gin@nat@height>\textheight\textheight\else\Gin@nat@height\fi}
\makeatother
% Scale images if necessary, so that they will not overflow the page
% margins by default, and it is still possible to overwrite the defaults
% using explicit options in \includegraphics[width, height, ...]{}
\setkeys{Gin}{width=\maxwidth,height=\maxheight,keepaspectratio}
% Set default figure placement to htbp
\makeatletter
\def\fps@figure{htbp}
\makeatother
\setlength{\emergencystretch}{3em} % prevent overfull lines
\providecommand{\tightlist}{%
  \setlength{\itemsep}{0pt}\setlength{\parskip}{0pt}}
\setcounter{secnumdepth}{-\maxdimen} % remove section numbering
\newlength{\cslhangindent}
\setlength{\cslhangindent}{1.5em}
\newlength{\csllabelwidth}
\setlength{\csllabelwidth}{3em}
\newlength{\cslentryspacingunit} % times entry-spacing
\setlength{\cslentryspacingunit}{\parskip}
\newenvironment{CSLReferences}[2] % #1 hanging-ident, #2 entry spacing
 {% don't indent paragraphs
  \setlength{\parindent}{0pt}
  % turn on hanging indent if param 1 is 1
  \ifodd #1
  \let\oldpar\par
  \def\par{\hangindent=\cslhangindent\oldpar}
  \fi
  % set entry spacing
  \setlength{\parskip}{#2\cslentryspacingunit}
 }%
 {}
\usepackage{calc}
\newcommand{\CSLBlock}[1]{#1\hfill\break}
\newcommand{\CSLLeftMargin}[1]{\parbox[t]{\csllabelwidth}{#1}}
\newcommand{\CSLRightInline}[1]{\parbox[t]{\linewidth - \csllabelwidth}{#1}\break}
\newcommand{\CSLIndent}[1]{\hspace{\cslhangindent}#1}
\usepackage{floatrow}
\floatsetup[figure]{capposition=top}
\usepackage{float}
\floatplacement{figure}{H}
\usepackage[T1]{fontenc}
\usepackage[utf8]{inputenc}
\providecommand{\keywords}[1]{\textbf{Keywords:} #1}

\providecommand{\wordcount}[1]{\text{Word Count:} #1}

\providecommand{\declaration}[1]{\text{}#1}

\usepackage{booktabs}
\usepackage{longtable}
\usepackage{array}
\usepackage{multirow}
\usepackage{wrapfig}
\usepackage{float}
\usepackage{colortbl}
\usepackage{pdflscape}
\usepackage{tabu}
\usepackage{threeparttable}
\usepackage{threeparttablex}
\usepackage[normalem]{ulem}
\usepackage{makecell}
\usepackage{xcolor}
\usepackage{siunitx}
\newcolumntype{d}{S[input-symbols = ()]}
\ifLuaTeX
  \usepackage{selnolig}  % disable illegal ligatures
\fi

\title{Performing Authenticity: Anti-Political Correctness as an
Authenticity Performances in Politics in Brazil and the United States}
\author{Henrique Sposito\\
\strut \\
Graduate Institute of International and Development Studies\\
Geneva, Switzerland\\
\href{mailto:henrique.sposito@graduateinstitute.ch}{\nolinkurl{henrique.sposito@graduateinstitute.ch}}}
\date{22 September 2022}

\begin{document}
\maketitle
\begin{abstract}
Various politicians have publicly denounced ``political correctness''
(PC). While some scholars argue that anti-PC discourses in politics are
populist tools to label elites or a form of cultural backlash against
liberal changes, these explanations focus on specific manifestations of
anti-PC by specific politicians. Instead, this article argues that
anti-PC discourses are performances of authenticity in politics. As
authenticity performances, anti-PC discourses reduce the perceived link
between thinking and saying for audiences. A framework for identifying
several discursive performances of authenticity in politics that focuses
on displays (what), projections (who, when, and where), and mechanisms
(how) is developed. A purpose-built dictionary of terms is designed to
detect authenticity performances in text datasets of campaign rallies,
debates, interviews, and official speeches gathered for presidents and
presidential candidates in Brazil and the United States since the 1980s.
The analysis indicates that authenticities are not performed more
frequently in election years, politicians usually perform different
authenticities most when they are candidates or after having left
office. Although, in the case of Brazil, authenticity performances
spiked from 2011 to 2016 while Dilma Rousseff was in office. The types
of authenticity performed over time also changed in each case across
time, indicating that politicians adapt to perform what audiences ``want
to hear''. Lastly, debates have become the setting in which authenticity
is performed most frequently, whereas interviews are the setting in
which authenticity is performed least frequently, in both cases in
recent years.
\end{abstract}

\keywords{authenticity, political correctness, performance, Brazil, United States, populism}

\wordcount{10000 (including references, footnotes and tables)}

\declaration{The author declares no competing interests.}

\hypertarget{introduction}{%
\section{1 Introduction}\label{introduction}}

In the second sentence of his inaugural speech as president of Brazil,
Jair Bolsonaro declared the day in which the people began to be free
from political correctness\footnote{ Find the whole speech
  \href{https://www.youtube.com/watch?v=5iPVlE_9kFw}{here}.}. Bolsonaro
is not unique in this sense, politicians as Trump, Sanders, Bush, and
Lula, have openly denounced political correctness (PC)\footnote{ PC is
  used as an abbreviation for political correctness and politically
  correct throughout the article, that is, as a noun and as an
  adjective.} and/or publicly employed politically incorrect language.
Broad audiences appear to respond positively to these statements in
multicultural countries, such as the United States (US) and Brazil,
where large portions of the population fall under racial, ethnic, and
other categories PC language attempts to safeguard\footnote{ Most
  Americans appear unsupportive of PC language with estimations that
  range from 52\% of the population (Montanaro 2018) to 80\% of the
  population disliking PC (Mounk 2018). In Brazil, 56 \% of participants
  in a national poll agreed completely or partially that the `PC patrol
  is making the world too boring' (Goncalves and Goncalves 2020).}.
Populist scholars argue that contemporary populists' profit from
breaking with PC language and use PC to identify modern elites (see
Mudde 2004). Cultural backlash scholars argue that anti-PC discourses
speak to resentment caused by silent liberal changes (see Norris and
Inglehart 2019). However, both populism and cultural backlash focus on
specific manifestations of these discourses by specific leaders that
embody their arguments, leaving aside any systematic or empirical
analysis of language usage or patterns across contexts. This article
argues that anti-PC discourses in politics are one performances of
authenticity. As an authenticity performance, anti-PC language reduces
the perceived link between thinking and saying to audiences (see Conway
et al. 2009; Conway, Repke, and Houck 2017). But there are also other
political performances that can generate an authenticity perception in
audiences. \emph{How, then, authenticity has been performed in politics
over time?}

This article develops a framework to identify various types of
authenticity performances in political discourses. Performances are the
projections of definitions of a situation to others (see Goffman 1956;
Alexander, Giesen, and Mast 2006). Performance allows to theorize that
audience's interpretation hinges on factors beyond discursive content or
meaning, such as how things are said. Authenticity, as a perceived
character trait that conveys ones' convictions, is connected to higher
levels of political trust from electorates (Stiers et al. 2021;
Valgarosson et al. 2021) and can be essential to a candidate's political
success (see Alexander 2010; Fordahl 2018). The framework focuses on the
performative display (what), the projection (who, when, and where), and
the mechanisms (how). Authenticity performances are divided into two
types, individual and collective. Individual authenticity performances
derive plausibility from audiences' expectations about a political
performer. These performances include claims of truth telling, lie
accusations, taking responsibility for actions, or pointing fingers at
other politicians' mistakes. Collective authenticity performances derive
plausibility from the shared cultural knowledge between politicians and
audiences. These include references to origins, allusions to common
sense, anti-PC discourses, or assertions of territorial knowledge.

Political discourses at the national level in Brazil and the US since
the 1980s are gathered and explored to assess variation in authenticity
performances over time, by politicians, and across different settings.
The 1980s were chosen as the starting point as they mark the beginning
of the debates surrounding PC language in the US and the decade in which
democracy was re-established in Brazil. Texts for campaign rallies,
debates, interviews, and official speeches for presidents and
presidential candidates in Brazil and the US were scraped to construct
the datasets. A purpose-built dictionary of terms was created to capture
various authenticity performances, as anti-PC, in discourses.

The findings indicate that the frequencies of authenticity performances
are not systematically greater in election years. Usually, politicians
perform authenticities above the 95th percentile when they are
candidates, before being elected a first time, or after having left
office. However, in the case of Brazil, we see a spike in the frequency
authenticity is performed in politics from 2011 to 2016, the Rousseff
years in office. Rousseff, arguably, performed authenticity to justify
herself and her public policy choices more frequently than others in the
sample. Moreover, the variation in types of authenticity performed over
time and across cases indicates that background representations might
make different performances more, or less, compelling at certain
junctures. Indeed, many authenticity performances appear in high
frequencies for opposing, and associated, candidates in similar years as
is the case of anti-PC performances in Brazil in the mid-1990s.
Politicians, thus, adapt to perform authenticities audiences ``want to
hear''. Finally, debates are the setting in which authenticity is
performed most frequently, whereas interviews are the setting in which
authenticity was performed the least frequently, in both cases in recent
years. Debates are large-scale media events that produce ``sticky''
political bites charged with imagery that circulate to mark election
cycles in these democracies. Whereas social media platforms give
politicians diverse outlets to interact directly with audiences,
bypassing journalists in interviews.

\hypertarget{review-anti-pc-politics}{%
\section{2 Review: anti-PC politics?}\label{review-anti-pc-politics}}

\hypertarget{pc-and-anti-pc-overview-and-effects}{%
\subsection{2.1 PC and anti-PC overview and
effects}\label{pc-and-anti-pc-overview-and-effects}}

Definitions of PC range from an ``idealistic intervention'' to ``liberal
fascism'' (Hughes 2011; Feldstein 1997). In practice, PC language avoids
judgmental terms, preferring euphemistic substitutions and it
presupposes that lexicon changes mediate discrimination in positive ways
(Hughes 2011, 13)\footnote{ For a good discussion of lexicalization and
  language change in linguistics, refer to Brinton and Traugott (2005).
  For a discussion on the complexities involved in why language changes,
  and not, as well as how meaning might be affected by such changes
  refer to Bybee (2015)}. Anti-PC discourses, instead, represent a
\emph{dismissal of PC substitutions and/or the denouncing of PC language
and users}. In order to understand what anti-PC language entails, we
must first briefly overview PC's history and effects. The modern
conception of PC originated in Mao Tsé-Tung's depiction of the `correct'
socialist party line in the 1930s, used to describe doing the right
things and thinking the right thoughts (Hughes 2011). The term was
picked up by leftist circles in the US during the 1960s and used to
describe more orthodox followers of socialism or as a critique of
excessive orthodoxy (Feldstein 1997; Weigel 2016; Hughes 2011). It was
not until the 1980s that right-leaning conservative elites in the US,
including many academics, started denouncing PC language substitutions
as a strategy to restrict freedoms of speech (Feldstein 1997; Weigel
2016). These conservative elites were able to swiftly recycle the
meaning of PC by disconnecting it from historical context, and
conflating it with enemy building narratives and imagery (Feldstein
1997).

The so-called ``university debates'' of the 1980s and 1990s across
American universities brought widespread attention to PC,
multiculturalism, and affirmative action (Berman 2011). On these
debates, D'Souza (1991) argues that the apparent duplicity between a
liberal educational system and multicultural claims makes PC language
further victimize minorities on campuses, while affirmative action
policies undermine their merits and can generate more racist backlash.
As a response, Gutmann and Habermas highlighted how misplaced PC debates
politicized and polarized important societal dialogues (Gutmann 1994).
Yet, most multiculturalism scholars at the time dealt with PC as a
matter of public or party disagreement and lamented their popularity
with certain communities while leaving unaddressed the nature of PC
language, the relation between the speaker's identity to the information
provided, and why diverse communities related or contested semantic
changes (Loury 1994). The university debates helped disseminate and
connect matters of PC, affirmative action policies, and multiculturalism
to left leaning (progressive) political ideologies and anti-PC to
right-wing (conservative) ideologies across the world (Feldstein 1997;
Bush 1995). By the mid-1990s, PC debates in the US had become as much
about rhetorical strategies to forward ideological political agendas as
about the diverse cultural movements' efforts to re-label (Feldstein
1997; Hall 1994).

The spreading of issues covered by PC language and the complex political
forces supporting, and opposing, it required that the cultural and
individual effects of PC in society be theorized. Hall (1994), for
example, describes the ambiguous truth and contested cultural authority
of the PC phenomenon to argue that individuals may disapprove of PC
language because these are relatively new demands for cultural
transformation. Similarly, Fairclough (2003, 24) argues that PC is a
socially constructed cultural and political label for which effects
depend on the resistance of structure and practices. Even though
euphemistic substitutions are not a modern manifestation, as changing
orthodoxies under moral imperatives exist since the invention of
printing, PC language has expanded the number of substitutions (Hughes
2011). This expansion of PC language generates more abstract and
imprecise replacements that can feel unnatural, create confusion,
patronize subjects, and further socioeconomic inequalities via
linguistic processes (Hughes 2011). This contributed to the evolution of
PC from a noun used to describe language substitution to an adjective
used to describe excess politeness or evasion of truths in society or
for individuals (Weigel 2016; Chait 2015).

\hypertarget{anti-pc-populisms-and-cultural-backlash}{%
\subsection{2.2 Anti-PC, populism(s), and cultural
backlash}\label{anti-pc-populisms-and-cultural-backlash}}

In \emph{The Populist Zeitgeist}, on the rise of populist parties in
liberal democracies, Mudde (2004, 543) defines populism as a thin
centered ideology\footnote{ To discuss whether, or the extent which,
  Bolsonaro or Trump are populists is beyond the scope of this article,
  for some discussions on this see Tamaki and Fuks (2020) and Hawkins
  and Rovira Kaltwasser (2018).}. The article peripherally discusses how
contemporary populists profit from breaking with politically correct
language, because citizens' increased emancipation made issues
surrounding PC more widespread, alongside how PC has been used to
identify a modern elite (Mudde 2004, 594--602). Since then, populists
have routinely been associated with anti-PC discourses. Still, Mudde's
analysis leaves undertheorized which elites are characterized as PC by
various populist leaders with a variety of ideological commitments. This
is puzzling since diverse political, economic, and intellectual elites
are often the drivers of both pro, and anti, PC discourses in societies.
Moreover, if populism is not a standalone ideology (see Mudde 2007), it
is hard to pinpoint if anti-PC discourses are a portion or manifestation
of the populist (thin)ideology, an adjacent ideology, or a contextual
feature of a specific society\footnote{ See Aslanidis (2016) for a good
  discussion on whether populism is an ideology.}. Besides, the account
leaves unclear when, and why, anti-PC discourses are a politically
profitable strategy for populists but not for other political actors.
Other influential theories of populism do not mention anti-PC discourses
(see Weyland 2001; Laclau 2005; Hawkins 2009) and, when mentioned,
anti-PC discourses fulfill a peripheral role to help identify
exclusionary right-wing populist parties in Europe (Betz 2001). In such,
there appears to be a leap associating populism with anti-PC discourses
without systematically or empirically analyzing its usage in time and
patterns across contexts.

At the intersection between populism, authoritarianism, and
multiculturalism, \emph{Cultural Backlash} argues that social
conservative individuals with authoritarian orientations react to some
trends with feelings of resentment for the erosion of respect for their
core values and that ``is the essence of the backlash against `political
correctness', in which sexist language, anti-foreigner sentiments, or
the expression of racist attitudes are condemned by the liberal
consensus and silenced in mainstream political debate'' (Norris and
Inglehart 2019, 123). Rather than developing further the mechanisms of
how anti-PC discourses work, deliberate on which types of anti-PC
discourses connect to resentment, the authors take for granted that
anti-PC discourses are connected to resentment, and that is why they
resonate with ``old, rural, or uneducated'' electorates\footnote{ Norris
  and Inglehart (2019) appear to `passively' pass on the blame to `old,
  rural, and uneducate' voters for `undesirable' political outcomes with
  broad causal explanations extrapolated from thin linear correlations
  that illustrate little of the complex relationships theorized. Others,
  as Mishra (2017), sometimes rely on a misplaced engagement with
  history that focus on a few instances that exemplify arguments as if
  these were the rule.}. Even so, anti-PC discourses appear to resonate
more broadly within societies for various reasons, many unrelated to
resentment (e.g.~humor)\footnote{ There are countless examples of PC and
  anti-PC language being explored in humoristic ways for entertainment
  as, for example, notoriously, TV shows as `South Park', `The Office',
  and `It's Always Sunny in Philadelphia'.}. How Bolsonaro, for example,
employs anti-PC in quick, direct, and short comments that focus on
culturally salient themes without being necessarily internally coherent
is, arguably, more decisive to his following than the content itself
(Carlo and Kamradt 2018). Or how Trump is characterized as `authentic',
as a character trait, appears more decisive to voters than his
recognizable inconsistencies or the negative meanings associated to what
he says (Fordahl 2018, 308)

Recent efforts to rethink populism as a political performance or a
repertoire focus on patterns of communication. Moffitt and Tormey (2014,
387) argue that we must rethink populism as ``the repertoires of
performance that are used to create political relations''. More
specifically, we must think about how populists perform ``the people'',
threats, breakdowns and crisis, and bad manners (Moffitt and Tormey
2014; Moffitt 2016). Concerning bad manners, the authors argue that
``much of populists' appeal comes from their disregard for `appropriate'
ways of acting in the political realm'' (Moffitt and Tormey 2014, 392).
Bad manners often performed with swearing, slang words, and political
incorrectness (Moffitt and Tormey 2014, 392). Likewise, arguing against
a pure definition of populism, Brubaker (2020, 60) contends that
populism is a discourse and stylistic repertoire that includes ``a
communication style that claims to favor plain-speaking, common sense
and authenticity against intellectualism and political correctness''. In
this sense, certain anti-PC discourses might be a portion of a broad
populist repertoire, although favoring plain speaking, displaying bad
manners, or being anti-PC are not exclusive to populists or connected to
populist discourses by ideology\footnote{ Brubaker (2017) view of
  anti-PC discourses is not inconsistent with Mudde (2004) point about
  PC and elites (see Brubaker 2020; Mudde 2007). Rather it implies that
  certain intellectual elites might be tagged as PC, not any ``modern''
  elite, and this is connected to anti-intellectualism. This is also
  consistent with Hughes (2011) argument about PC language expansion
  that leaves some people feeling patronized and confused by PC
  language.}. Both the cultural backlash and populism accounts of
anti-PC in politics focus on a few specific manifestations of these
discourses by particular leaders that embody their respective arguments.
Consequently, these literatures only partially explain what certain
discourses as anti-PC discourses are, how they function, or why they
might matter for political outcomes.

\hypertarget{theory-performing-authenticity}{%
\section{3 Theory: performing
authenticity}\label{theory-performing-authenticity}}

\hypertarget{performance-authenticity-and-anti-pc-in-politics}{%
\subsection{3.1 Performance, authenticity, and anti-PC in
politics}\label{performance-authenticity-and-anti-pc-in-politics}}

Performances are the projections of a situation when one appears before
others, ``however passive their role may seem to be, will themselves
effectively project a definition of the situation by virtue of their
response to the individual'' (Goffman 1956, 3). But why should we
understand discursive politics through performance or, rather, why do
politicians say the things they do? Conventionally, there are two
answers to this question. On the one hand, politicians might say the
things they do according to what is more profitable to them
(i.e.~rational choice). On the other hand, politicians might say the
things they do because of their beliefs (i.e.~ideology). Though, neither
may hold in practice as politicians can, at different times, say what is
more profitable, say what they believe in, or say things without
intentionally thinking about it. Normally, whether a politician is
saying what they believe in, or what is more profitable, depends on the
audiences' interpretation. Understanding politics through performance
offers a more flexible answer to why politicians say the things they do:
to project their understandings. Performance allows to theorize that
audience's interpretation hinges on factors beyond discursive content or
an interpretation of message meaning, such as how things are said (see
Alexander, Giesen, and Mast 2006; Alexander 2011). Seeing discursive
politics through performance provides more realistic assumptions to
theorizing how politicians ``do politics'' (see Van Dijk et al. 1997).

Capturing audience's perception and other discursive elements beyond
text is challenging. There are, however, several capturable discursive
displays connected to the how, the where, and the when of political
performances that allow to identify patterns and variation in political
performances as the performer's role, the script, the stage, and the
audience, rather than the meanings associated with the content of a
political discourse. By facilitating comparison, a focus on perormative
patterns provides more complete picture of how discursive politics
change, and not, in time and by politicians. There are also important
implications of thinking discursive politics through performances.
Performance places agency both with audiences, watching and evaluating
politicians ``doing'' politics, and with political performers (see
Alexander, Giesen, and Mast 2006, 35). This means script changes can be
theorized to be intentional individual innovations or unintentional
chattering by performers, while political accomplishments reflect
positive evaluations from audiences. This does not mean factors such as
social media, economic crisis, and cultural changes, among others, are
irrelevant. Instead, these become collective background representations
to be explored in political scripts and provide context for audiences'
interpretation (see Alexander, Giesen, and Mast 2006, 46; Alexander
2011).

Authenticity has long been discursively performed in politics with
self-references to origins, remarkable stories, allusions to civic
tradition, and displays of `vulgarism' (Fordahl 2018, 309; Alexander
2010). Authenticity is an individualistic performance that aims at
radiating truthfulness outwards (Taylor 1992; Fordahl 2018).
Authenticity in politics does not concern being truthful to oneself but
appearing coherent with individual or societal values to audiences
(Valgarosson et al. 2021; Fordahl 2018). Stiers et al. (2021) refer to
authenticity in politics as an essential character trait that conveys
candidates' convictions and that has positively affected electoral
preferences across several European democracies. As a political tool,
authenticity helps build political trust for candidates by demonstrating
to electorates that politicians are in touch with ordinary people and
their struggles (Valgarosson et al. 2021). Authenticity perceptions
were, for example, essential for Obama's success in 2008 (Alexander
2010) and, arguably, defined the 2016 American election in Trump's favor
(Fordahl 2018). Authenticity, in politics, is an unstable and malleable
label that requires social validation while demanding contortion,
modification, and active individual effort (Fordahl 2018)\footnote{
  Authenticity is central to Alexander, Giesen, and Mast (2006) approach
  to social performance (see Alexander 2011). Authenticity for the
  author is both an attribution and the measure of performative success
  (Alexander, Giesen, and Mast 2006, 55). Authenticity here is a
  performance that attempts to radiate truthulness, even if for some it
  does and others not.}.

As a communication norm, PC language attempts to remove negative
language by means of self and group censorship (Conway et al. 2009;
Conway, Repke, and Houck 2017). This communication norm can backfire due
to contamination of individual information processes by authority or
legitimacy effects (Conway et al. 2009; Conway, Repke, and Houck 2017).
Authority effects surrounding PC language occur when language is
believed to be insincere if it is commanded by an authority (Conway et
al. 2009). Legitimacy effects are caused by individuals' own
self-censorship, which leads to increased awareness that language
interactions may not be genuine (Conway et al. 2009). Building on these
findings, Rosenblum, Schroeder, and Gino (2020) use several survey
experiments to illustrate how politically incorrect language makes
political communicators appear more authentic, have stronger
convictions, and be less strategic, in comparison to PC politicians. In
fact, the denouncing of a ``PC politician'' or a ``PC ideology''
engrains an allusion to inauthenticity, to someone or something that
expresses its views in calculated ways to avoid judgment (Hughes 2011;
Weigel 2016). Anti-PC discourses are connected to authenticity in
politics by reducing the perceived link between thinking and saying for
audiences (see Conway, Repke, and Houck 2017)\footnote{ The argument
  also considers the insights of bad manners as a repertoire (Moffitt
  2016; Moffitt and Tormey 2014) of plain speaking as communication
  style (Brubaker 2020, 2017) in populism. Though the conceptualization
  here is broader and intends to theorize anti-PC outside of populism,
  it is somewhat consistent with how these theorists understand what
  these repertoires/styles are, how they work in politics, and why they
  might matter.}. Anti-PC performances signal to audiences' precisely
that politicians might be coherent to an inner self. Though,
authenticity in politics can be performed in politics in several other
ways.

\hypertarget{authenticity-performances}{%
\subsection{3.2 Authenticity
performances}\label{authenticity-performances}}

Stiers et al. (2021) associate authenticity in politics to a sense of
``realness'' that, in recent times, has repeatedly turned to
anti-politics to connect politicians with ordinary people. The authentic
politician is an abstract individual trait or group perception, with
broad political implications. How does authenticity come about in
politics? Looking at authenticity in politics, as a performance,
requires identifying displays, projections, and mechanisms. Display
concerns detecting a certain performance (what) and is constrained by
projection and mechanism. Projection relates to the level at which it is
convincing that a performance is authentic. Projection encapsulates the
role (who), the setting (where), and the structure (when) for certain
display(s). The role relates to the part a performer takes in politics.
Roles can generate different expectations and ranges of possibilities
for authenticity performances (e.g.~candidates versus elected
officials). Setting refers to where a performance takes place
(e.g.~debate or official speech). Structure indicates the timing in
which performance is inserted (e.g.~before/after an election).
Mechanisms refer to the theorized pathways (how) by which a projected
display might work to produce authenticity. Diverse answers to each of
these performative aspects generate different expectations about
performances of authenticity in politics.

Although what could be an authenticity performance is not always
straight forward or capturable discursively as the ability of
politicians to ``fuse'' displays, projection, and mechanisms to radiate
truthfulness is bounded by audience's interpretation (see Alexander,
Giesen, and Mast 2006), there are several indications of how
authenticity has concretely been performed discursively in politics.
Alexander (2010), for example, discusses how allusions to origins,
territory, and civic traditions were employed by Obama to connect with
audiences and generate a sense of authenticity. These performances focus
on the cultural connections shared by politicians and audiences and are
essential to legitimize a candidate's knowledge about the ``real''
issues people in the country face. Fordahl (2018) argues that Trump's
authenticity was built upon iconic, often vulgar, representations of
American traditions and reality performed consistently from his
``straight shooter dealmaker'' role. These noticeable types of
performances in discursive politics can be understood as authenticity
performances.

Authenticity performances can be divided in two types: \emph{individual}
and \emph{collective}. The types refer to the mechanism that might give
plausibility to a certain performance. Individual authenticity
performances derive plausibility for performance based on audiences'
expectations about a political performer (or opponent) considering the
information they have. These performances include stating to be telling
the truth, claiming others are lying, taking responsibility for actions,
or pointing fingers at other's errors. Collective authenticity
performances are more elaborate displays of authenticity that derive
plausibility for performance based on the cultural connections shared
between audiences and performer. These performances include
anti-PC\footnote{ Politically incorrect expressions coded in dictionary
  were selected from a 1992 dictionary of politically correct language,
  this assumes that most of these terms have for long been agreed upon
  as not PC (Beard and Cerf 1993).}, pointing at origins, allusions to
common sense, or claims of territorial knowledge. Table 1, below,
summarizes each authenticity performance theorized, the type, their
respective displays, and mechanisms.

\begin{landscape}

\begingroup\fontsize{10}{12}\selectfont

\begin{longtabu} to \linewidth {>{\raggedright\arraybackslash}p{3.5 cm}>{\raggedright\arraybackslash}p{3 cm}>{\raggedright\arraybackslash}p{10 cm}>{\raggedright\arraybackslash}p{4 cm}}
\caption{\label{tab:Table 1}Authenticity Performances, Displays, and Mechanisms}\\
\toprule
Authenticity Performance & Type & Displays & Mechanism\\
\midrule
\endfirsthead
\caption[]{Authenticity Performances, Displays, and Mechanisms \textit{(continued)}}\\
\toprule
Authenticity Performance & Type & Displays & Mechanism\\
\midrule
\endhead

\endfoot
\bottomrule
\endlastfoot
\textbf{\cellcolor{gray!6}{Truth Telling}} & \cellcolor{gray!6}{Individual} & \cellcolor{gray!6}{Mentions truthfulness, sincerity, and honesty when describing oneself. Examples: 'the truth is', 'this is the truth', 'am not lying', 'is/are/am honest', 'honesty', 'is/are/am sincere', 'is/are true'} & \cellcolor{gray!6}{Speaker appears to be telling the truth regarding their beliefs or facts}\\
\textbf{Lie Accusations} & Individual & Mentions dishonesty, untruthfulness, and insincerity when used to describe others. Examples: 'not the truth', 'not true', 'untruthful', 'is/are lying', 'is/are liars', 'is/are dishonest', 'is/are fake', 'is/are a hypocrite' & Speaker appears more sincere vis-a-vis others\\
\textbf{\cellcolor{gray!6}{Consistency}} & \cellcolor{gray!6}{Individual} & \cellcolor{gray!6}{Mentions career consistency, responsibility, accountability for individual. Examples: 'I/we delivered', 'check and see', 'keep my word', 'keep promises', 'I am/we are responsible', 'I/we take responsibility', 'I/we guarantee'} & \cellcolor{gray!6}{Speaker appears consistent regarding pledges}\\
\textbf{Finger Pointing} & Individual & Mentions lack of accountability, inconsistency and/or blame others for mistakes. Examples: 'are/is inconsistent', 'are/is irresponsible', 'their fault', 'not my fault', 'they left us with', 'are/is responsible', 'costed us', 'false/fake/broken promises' & Speaker appears not accountable for previous undesirable outcomes\\
\textbf{\cellcolor{gray!6}{Origins}} & \cellcolor{gray!6}{Collective} & \cellcolor{gray!6}{Alludes to birthplace, origins, and roots to describe background, values, and tell their story. Examples: 'I was born in', 'I come from', 'I/we grew up in', 'my family', 'I was raised', 'my background', 'I was thought', 'my hometown/community/city'} & \cellcolor{gray!6}{Speaker seems culturally connected to the nation}\\
\addlinespace
\textbf{Common Sense} & Collective & Alludes to common sense, reason, and logic to describe choices or preferences. Examples: 'is/are common sense', 'everyone/everybody knows', 'it is undeniable', 'stating the obvious', 'everyone agrees', 'we all know', 'no one disagrees that', 'we have all learned that' & Speaker seems to make choices consistent with what others in society would do\\
\textbf{\cellcolor{gray!6}{Territory}} & \cellcolor{gray!6}{Collective} & \cellcolor{gray!6}{Alludes to sub-portions of the territory known and/or visited. Examples: 'have seen in', 'have been to', 'I/we/have visited', 'came all the way to', 'came/came back from', 'saw/see first-hand in', 'I/we were hosted', 'my/our time in'} & \cellcolor{gray!6}{Speaker seems territorially connected to sub-regions, regions, or nation}\\
\textbf{Anti-PC} & Collective & Alludes to PC language negatively, employs politically incorrect language, or claims to speak what one thinks without filters. Examples: 'not politically correct', 'political correctness', 'speak/speaking my mind', 'say/saying what everyone thinks', 'colored/oriental/fat/handicapped people' & Speaker seems to be saying what they think on culturally contested themes\\*
\end{longtabu}
\endgroup{}

\end{landscape}

\hypertarget{methodology}{%
\section{4 Methodology}\label{methodology}}

\hypertarget{case-selection}{%
\subsection{4.1 Case selection}\label{case-selection}}

In Brazil, restrictions to freedoms of speech based on racism were
criminalized in the constitution of 1988 and affirmative action policies
in universities were put in the late-1990s. Although these were widely
discussed in society, albeit a long history of direct and indirect
political opposition to them, neither constitutional nor affirmative
action debates sparked much discussion about PC (see Farias 2001;
Freitas and Castro 2013). In fact, the PC terminology was translated
from outside Brazil (Possenti 1995) and it did not directly appear in
public debates until 2004, when the Secretariat of Human Rights of the
President published a manual of PC language which contained 96 popular
expressions deemed incorrect for being racist, sexist, or homophobic
(Fiorin 2008; Morato and Bentes 2017). A that time, the federal
government quickly recalled the publication due to internal opposition
and external backlash (Fiorin 2008; Morato and Bentes 2017). Since the
mid-2000s humoristic interpretations of politically incorrect language
and politically charged calls against a PC country started to appear
frequently in the Brazilian media (Morato and Bentes 2017; Weinmann and
Culau 2014). Alongside this, several articles in popular periodicals and
popular books were written opposing PC language (Weinmann and Culau
2014). By the 2010s, PC started to be referred to as a ``controversy''
and, nowadays, PC in Brazil is a widely contested phenomenon in society
and politics.

Apart from that, Brazil and the US held the world's largest slave
populations until slavery was abolished respectively in each country.
Similarly, the two countries became large settler states for diverse
groups of migrants in the following centuries. These historical
processes rendered their current populations heterogenous in terms of
demographic composition and cultural heritage. This explain the
appearance and influence of PC language in these societies, but the
eminence of anti-PC discourses in national politics is somewhat puzzling
as electorates are likely to encompass several large groups that PC
language safeguards. While Brazil and the US are relatively similar in
geographical and population sizes, they are dissimilar in many other
ways including levels of economic development, culture and, in some
ways, their political systems. Thus, each case is understood as a
configuration, formed by the aggregation of parts that make sense in the
context of each case (Ragin 1987).

\hypertarget{data}{%
\subsection{4.2 Data}\label{data}}

Text data on official speeches, campaign rallies, debates, and
interviews for elected presidents and runoff candidates were gathered
since 1980 for the US and 1985 for Brazil \footnote{For the US, runoff
  candidates are the nominated democratic and republican presidential
  candidates. For Brazil, these are the candidates that went on to the
  second round of presidential election. When election was decided in
  the first round, the two leading candidates were selected.}. In the
US, the 1980s marked the beginning of the PC debates. In Brazil,
democracy came back in 1985, though direct presidential elections were
only held in 1989. All the texts for the US were scraped from
\href{https://www.presidency.ucsb.edu}{The American Presidency Project}
repository (TAPP). Collecting data for Brazil proved more difficult due
to availability. For official speeches, Cezar (2020) dataset on official
speeches for Brazilian presidents from 1985 to 2019 was updated to
include the missing speeches from 2019 to 2021 using the
\href{http://www.biblioteca.presidencia.gov.br/sobre-a-biblioteca/biblioteca-da-presidencia-da-republica}{Brazilian
Presidential Library}. Text for debates, interviews, and campaign
rallies were scraped from subtitles automatically generated for YouTube
videos. The number of videos available for later election cycles, after
the 2000s, is considerably larger than earlier ones. Additionally, some
election cycles in Brazil were shorter than others as national elections
were decided on the first round, therefore, data for those cycles is
more limited. For these reasons, and due to the longer time scope, the
number of observations in the text datasets for the US is greater than
for Brazil\footnote{ For all the data, scripts, and additional
  information please request access to the authenticity performances
  repository available on GitHub. For more information on some the
  functions developed see the poldis (Sposito 2021) R package.}.

The comparison of data for different settings in which politics gets
done provide more complete picture of how discursive politics change,
and not, in each of these settings, across time, or by politicians.
Table 2, below, summarizes the number of text observations by setting in
each case, the earliest date, the latest date, and the sources. For
interviews, those occurring in the period between two years before
elections and one year after the election (or tenure for elected) were
compiled. For debates, data on runoff debates in Brazil and debates
after party candidates were nominated in US were gathered\footnote{
  Those are debates after party nominations in the US and runoff debates
  for Brazil. There are a few exceptions to this in Brazil for elections
  decided in the first round (e.g.~1994) or elections where candidates
  were unable to participate in runoff debates (e.g.~2018). In these
  cases, the participations of the two most voted candidates in the
  first-round debates were gathered. The number of debates for both
  Brazil and the US in Table 2 reflect the number of politicians coded
  for each runoff debate in dataset. That is, the text of each debate
  was separated by politicians.}. Campaign remarks occurring on election
years for runner-up candidates were gathered. Therefore, the datasets
for campaign and debates have their latest date as the last election
year in each case, that is 2018 for Brazil and 2020 for the US.

\begin{table}[H]

\caption{\label{tab:table 2}Text Data for Brazil and the US}
\centering
\begin{tabular}[t]{>{}llrrr}
\toprule
Country & Setting & Observations & Earliest Date & Latest Date\\
\midrule
\textbf{US} & Speeches & 15016 & 1981 & 2021\\
\textbf{US} & Campaign & 1563 & 1980 & 2020\\
\textbf{US} & Debates & 59 & 1980 & 2020\\
\textbf{US} & Interviews & 936 & 1980 & 2021\\
\textbf{Brazil} & Speeches & 6130 & 1985 & 2021\\
\addlinespace
\textbf{Brazil} & Campaign & 175 & 1989 & 2018\\
\textbf{Brazil} & Interviews & 262 & 1987 & 2021\\
\textbf{Brazil} & Debates & 29 & 1989 & 2018\\
\bottomrule
\end{tabular}
\end{table}

\hypertarget{operationalization}{%
\subsection{4.3 Operationalization}\label{operationalization}}

After collection, texts were cleaned by removing punctuations and
accents. Authenticity performances, including anti-PC discourses, were
identified via a purpose-built dictionary of terms that codes the
discursive displays associated with each performance (see Codebook in
Appendix). The dictionary of terms was developed listening to samples of
randomly selected speeches, campaign remarks, interviews, and debates
from the datasets. The dictionary has similar definitions in Portuguese
and English in relation to the words and expressions searched. The
number of words included in the dictionary for each performance is
similar across languages. The dictionary was designed to reduce the
possibility of overlaps, even as some authenticity performances might
share similar displays. Directionality in the text is important to
identify authenticity performances and to distinct when speakers talk
about themselves or others, thus, no stop words were removed from texts.
This means the dictionary of terms includes combinations of
pronouns/determiners and verbs/nouns to avoid false-positive matches.
All frequencies of authenticities performances are normalized for the
number of words in each text. That is, the number of matches for
authenticity performances in each text was divided by the number of
words in the text they appear in. All the normalized scores were then
multiplied by one thousand to improve visualizations, therefore,
represent the rates for every 1000 words. Normalization helps account
for discrepancies in the number of observations for the two cases, and
for the same case across time, to facilitate comparison.

\hypertarget{analysis-say-it-like-you-mean-it}{%
\section{5 Analysis: say it like you mean
it}\label{analysis-say-it-like-you-mean-it}}

\hypertarget{authenticity-performances-in-time-in-brazil-and-the-us}{%
\subsection{5.1 Authenticity Performances in Time in Brazil and the
US}\label{authenticity-performances-in-time-in-brazil-and-the-us}}

\begin{table}[H]

\caption{\label{tab:table 3}Average Normalized Proportion of Authenticity Performances in Brazil and the US (for every 1000 words)}
\centering
\begin{tabular}[t]{>{}lll}
\toprule
Authenticity Performance & Average US & Average Brazil\\
\midrule
\textbf{Truth Telling} & 0.33 & 0.35\\
\textbf{Lie Accusations} & 0.04 & 0.03\\
\textbf{Consistency} & 0.19 & 0.20\\
\textbf{Finger Pointing} & 0.02 & 0.06\\
\textbf{Origins} & 0.34 & 0.85\\
\addlinespace
\textbf{Common Sense} & 0.06 & 0.25\\
\textbf{Territorial} & 0.10 & 0.05\\
\textbf{Anti-PC} & 0.09 & 0.03\\
\bottomrule
\end{tabular}
\end{table}

Allusions to origins and claims of truth telling are the two most
regularly performed authenticities in both Brazil and the US. Table 3,
above, displays the average (normalized) proportions for each
authenticity performance by country for every 1000 words. While origins
are a collective authenticity performance (i.e.~based on the cultural
connections shared between audiences and performer), truth-telling is an
individual performance (i.e.~based on audiences' expectations about a
political performer). Yet, as authenticity performances, they promote
oneself instead of focusing on others. Unsurprisingly, politicians speak
mostly about themselves when doing politics. Indeed, authenticity
performance that focus on others, such as lie-accusations and
finger-pointing, are performed infrequently, on average, in both cases.
Consistency, an inward-looking individual authenticity performance, also
appears relatively frequently in both cases. In the case of Brazil,
common sense is also performed relatively frequently.

How, though, do the frequencies of authenticity performances change over
time? Figure 1, below, illustrates the total frequencies for
authenticity performances in Brazil and the US in time. In the figure,
the x-axis represents the years and the y-axis represents the normalized
sum of authenticity performances. The black dots represent election
years for each case. While appearing to electorates is shown to
influence election outcomes (Fordahl 2018; Stiers et al. 2021;
Valgarosson et al. 2021), the figure shows no systematic increases in
the total frequency scores associated to election years in Brazil or the
US over time\footnote{ A linear regression that correlates election
  years, as factors, to the total of authenticity performances for each
  case has also been run, see Table 4 in Appendix. The coefficients show
  that for both cases the relationship between election years and the
  frequency of authenticity performances is slightly negative
  (i.e.~elections year correlate with a decrease in the total of
  authenticity performances). This relationship is negative and
  statistically significant for the case of the US.}. This indicates
that politicians might be more generally careful towards when, where,
and how authenticity is performed around election years, political
discourses become more instrumental and less improvised.

\begin{figure}
\centering
\includegraphics{ap_sposito_files/figure-latex/Figure 1-1.pdf}
\caption{Authenticity Performances in Time in Brazil and the US}
\end{figure}

Authenticity has generally been performed with greater frequency in
politics in Brazil since the mid-1990s in comparison to the US, which
could be related to the differences in the number of parties in the
political system. Brazil features extreme party fragmentation and weak
partisanship (Baker, Ames, and Rennó 2020), thus, politicians are less
susceptible to broad ideological pressure to conform and represent
interests of heterogenous groups within it, in comparison the US with
two major parties. We also see a large increase in frequencies of
authenticities performed in Brazil between 2011 and 2016, the Rousseff
years\footnote{ A linear regression correlating the total of
  authenticity performances to years, as factors, in Brazil confirms
  statistically significant increases for the years of 2012 and 2015,
  non-election years and when Rousseff was president, see table 6 in
  Appendix.}. This might indicate a relationship between gender and the
frequencies at which authenticity is performed in politics. However,
there are only two women politicians in the sample (Rousseff and H.
Clinton) which makes it challenging to extrapolate how gender and
authenticity performances correlate, even if women in politics have a
distinct communication style (see Wood 1994; Christine Banwart and
McKinney 2005; Blankenship and Robson 1995; Franceschet, Piscopo, and
Thomas 2016). Additionally, Rousseff is the only non-professional
politician elected president of Brazil. Having held several
techno-bureaucratic positions, Rousseff ran for office for the first
time in 2010 being elected president. Rousseff, arguably, needed to
justify herself and her public policies performing authenticities to
connect with audiences more frequently than others in th sample.

The types of authenticity performed in politics also change over time.
Figure 2, below, illustrates how collective and individual authenticity
performances have changed in time for Brazil and the US. In the figure,
the x-axis represents the years and the y-axis represents the normalized
values of individual and collective authenticity performances. In
Brazil, collective authenticity performances were performed considerably
more frequently than individual performances from the 1980s until the
early-2010s. This trend began to change at that point and, by 2019, we
see a reversal of this pattern whereas individual authenticity
performances surpass collective performances. Conversely, in the case of
the US, individual authenticity performances were performed more
frequently throughout the 1980s. From the early-1990s until the
early-2010s, both performances appear at very similar rates in the US.
This pattern changed in the mid-2010s with collective authenticity
performances surpassing individual. In both cases, this indicates that
background representations make some authenticity performances more, or
less, compelling to audiences at certain juctures. Audiences in Brazil,
for instance, could have became tired of the common collective
authenticity performances that focused on share cultural links with the
politico-economic crisis of the mid-2010s and, in turn, became more
interested in simpler individual performances that were credible based
on expectations about political performers. In the US, instead,
audiences arguably grew tired of balanced individual and collective
performances in favor performers and/or performances that focused on
shared cultural connections in the mid-2010s. In both cases, these
patterns began to change before most recent elections (2016 for the US
and 2018 for Brazil) and continue after even as politicians at the top
might have changed. This indicates that timing could favor certain
candidates who already perform certain authenticities and that
politicians adapt to perform authenticities that audiences ``want to
hear''.

\begin{figure}
\centering
\includegraphics{ap_sposito_files/figure-latex/Figure 2-1.pdf}
\caption{Collective versus Individual Authenticity Performances in Time
in Brazil and the US}
\end{figure}

\hypertarget{authenticity-performances-by-politicians-in-brazil-and-the-us}{%
\subsection{5.2 Authenticity Performances by Politicians in Brazil and
the
US}\label{authenticity-performances-by-politicians-in-brazil-and-the-us}}

Presidents and presidential candidates focus on some authenticity
performances over others. Figure 3, below, captures authenticity
performances by presidents and presidential candidates that fall above
the 95th percentile in a certain year. In the figure, the x-axis
represents the years and the dots represent a politician that performed
an authenticity above the 95th. The 95th percentiles are calculated for
each authenticity performance and for the total of authenticity
performances. Most politicians in Brazil and the US performed one, or
more, authenticities above the 95th percentile when they were
candidates, before being elected the first time (e.g.~Lula), or after
having left office (e.g.~Clinton). Other politicians do not perform any
authenticity at the 95th percentile for any year campaigning or in
office (e.g.~Reagan and Cardoso). This indicates that these politicians
either perform authenticity less than others in general and/or that they
do not focus on one authenticity more than other politicians at any
point in time\footnote{ Since authenticity performances as lie
  acusations and figer pointing happen, on average, very infrequently
  the 95th percentile for these are not included on Figure 3 to improve
  visualization. As well, some outlier scores were made smaller to fit
  the visualization. These are Modale in 1983 (Total = 4.2) and Lula in
  1992 (Origins = 5.5 and Total = 7.4).}. In the case of the US,
Mondale, Clinton, Gore, Obama, and H. Clinton perform the total of
authenticity performances above the 95th percentile at a certain year,
but none of these politicians were in the oval office when they did so.
In the case of Brazil, Lula, Collor, Neves, and Rousseff performed the
total of authenticities above the 95th percentile at a certain year, but
only Roussef did so while president. Rousseff performed common sense,
consistency, and origins above the 95th percentile during her years in
office, performances that are consistent with Wood (1994) (p.~137-148)
argument that women's style in politics include personal disclosures of
details, use of anecdotes, and concrete reasoning.

\begin{figure}
\centering
\includegraphics{ap_sposito_files/figure-latex/Figure 3-1.pdf}
\caption{95th Percentile Authenticity Performances by Politicians in
Brazil and the US}
\end{figure}

Associated and opposing politicians often perform the same
authenticities in similar years. Take, for example, the case of anti-PC
in Brazil which was performed by diverse politicians above the 95th
percentile around the mid-1990s. Anti-PC in the US, however, was
performed by Mondale in the 1980s and Trump from 2015
onwards\footnote{Though Trump and Mondale politicians belonged to
  different parties and had divergent political stands, both often
  employed a ``telling like it is'' communication style. For more on
  Modale's ``truth'' telling style see his
  \href{https://www.nytimes.com/1984/07/20/us/transcript-of-mondale-address-accepting-party-nomination.html}{1984}
  Democratic Convention acceptance speech. Some of Mondale's mentions of
  truth telling regarding raising taxes, for example, is similar to
  numerous accounts of Trump's denouncing PC in terms of wasting
  peoples' time.}. Indeed, Trump is the only politician the sample who
appears to consistently perform authenticity with anti-PC. Other
politicians in both cases, usually, perform different authenticities
above the 95th percentile over time. Furthermore, in the case of the US,
we also see that truth telling was performed above the 95th percentile
by politicians as Kerry and McCain around the year 2000; consistency was
performed above the 95th percentile in the 2000s by W. Bush, Clinton,
McCain; while origins was performed above the 95th percentile by Mondale
and Bush in the mid-1980s and by Gore and Clinton in the mid-1990s.
Comparably, in the case of Brazil, truth telling, origins, and
consistency were performed above the 95th percentile by Lula and Collor
from the late-1980s to the late-1990s. These similar performances above
the 95th percentile for associated and opposing politicians in similar
years indicate that politicians might respond to, or imitate, each other
on specific authenticity performances.

\hypertarget{authenticity-performances-across-settings-in-brazil-and-the-us}{%
\subsection{5.3 Authenticity Performances across settings in Brazil and
the
US}\label{authenticity-performances-across-settings-in-brazil-and-the-us}}

The frequency of authenticity performances changes across settings in
which politics gets done. Figure 4, below, illustrates authenticity
performances in Brazil and the US across setting over time. The x-axis
represents the years and the y-axis represents the normalized
authenticity performances for each setting. We see in the plot that, in
the US, authenticity was performed much more frequently in campaign
rallies than in all other settings until mid-2000s\footnote{ The
  relationship between the average frequencies of authenticity
  performances per year and setting was also investigated employing
  fixed-effects linear panel models. Fixed-effects models account for
  time effects while controlling for unobserved associations within the
  model variables (Allison 2009). In the regression (see Table 6 in
  Appendix), the correlation between the frequencies of authenticity
  performances and campaign settings for the US, in comparison to
  official speeches (reference category), is positive and highly
  statistically significant. Interviews also appear to correlate
  positively with authenticity performances in the US, in comparison to
  official speeches. In the case of Brazil, both campaign and debate
  settings correlate positively with authenticity performances in
  relation to official speeches. However, using this approach, we miss
  how these correlations change in time.}. At that point, the frequency
in which authenticity was performed across all settings in the US
becomes similar until the early-2010s, when performances in debates
increase and performances on interviews decrease in frequency. In the
case of Brazil, we see a different trend. The frequency at which
authenticity is performed in campaign rallies generally increased from
the late 1980s to the mid-2010s. In fact, there is a sharp increase in
the frequency in which authenticity is performed in campaign rallies,
debates, and official speeches form the late 2000s to the mid-2010s. We
then see a sharp decrease in the frequency in which authenticity is
performed across all setting from the mid-2010s onwards in Brazil.

\begin{figure}
\centering
\includegraphics{ap_sposito_files/figure-latex/Figure 4-1.pdf}
\caption{Authenticity Performances by Setting Time in Brazil and the US}
\end{figure}

In both Brazil and the US debates have become the setting in which
authenticity is performed most frequently, whereas interviews are the
setting in which authenticity is performed least frequently, from the
mid-2010s onwards. In the case of interviews, the spread of social media
gave politicians alternative outlets to interact directly with
audiences, bypassing journalists and their filters (see Alexander 2011,
106) while performing authenticity directly to wide portions of the
electorate. Relatedly, in terms of authenticity performances, debates'
format requires candidates to answer quick to sometimes unpredictable
questions and, as large-scale media events, become sources of ``sticky''
sound, text, and video bites charged with imagery, rather than meaning,
that circulate to mark and represent political cycles in democracies
(Foley 2012,; Coleman 2000)\footnote{Though why and the extent to which
  debates might matter for election outcomes is contentious (see
  McKinney and Warner 2013).}. As social media reduces the length of
political bites, debates, or rather the bites that it generates, become
ever an important settings for a wide variety of authenticity
performances.

\hypertarget{conclusion}{%
\section{6 Conclusion}\label{conclusion}}

This article set out to investigate how anti-PC discourses appear and
change over time in politics. A short dive into the history of PC, and
anti-PC, illustrated expansion of PC language generates more abstract
and imprecise replacements that can feel unnatural, create confusion,
patronize subjects, and further socioeconomic inequalities via
linguistic processes, while contributing to the evolution of PC from a
noun used to describe language substitution to an adjective used to
describe evasion of truths. A review of the populism and cultural
backlash scholarship illustrated how these literatures focus on
particular manifestations of anti-PC discourses by specific leaders.
Instead, this article argues that anti-PC discourses in politics are
authenticity performances that reduce the perceived link between what
politicians are thinking and what they are saying to audiences. Several
other authenticity performances are also theorized. A framework for
investigating authenticity performances in politics that focuses on
performative displays (what), projections (who, when, and where), and
mechanisms (how) is developed. Authenticity performances can be
individual and collective. Individual authenticity performances derive
plausibility from audiences' expectations about a political performer
(or opponent) considering the information they have. These performances
include claims of truth telling or consistency and lie accusations or
finger pointing. Collective authenticity performances derive
plausibility for performance based on the cultural connections shared
between audiences and performer. These performances include anti-PC,
pointing at origins, allusions to common sense, or claims of territorial
knowledge. A dictionary of terms for investigating authenticity
performances in discourses is built. Texts for campaign rallies,
debates, interviews, and official speeches for presidents and
presidential candidates in Brazil and the US since the 1980s were
scraped to construct the text datasets. This allowed for the
identification, comparison, and analysis of how authenticity
performances change over time, by politicians, and across political
settings.

The analysis of the findings reveal that authenticity performances that
promote oneself, as origins and truth-telling, occur with greater
frequency on average than other performances. As well, the frequencies
of authenticity performances are not systematically greater in election
years in comparison to non-election years. This indicates that
incumbents might be more careful towards when, where, and how
authenticity is performed around election years. Indeed, most
politicians perform one or more authenticities above the 95th percentile
when they are candidates, before being elected a first time, or after
having left office. However, in the case of Brazil, we see a spike in
the frequency authenticity is performed in politics from 2011 to 2016,
the Rousseff years. Rousseff performed authenticity more frequently to
justify herself and her public policy choices than others, arguably, due
to her gender and/or the fact that she was a relatively inexperienced
politician at the time. Moreover, the variation in types of authenticity
performed over time and across cases indicate that background context
make some types of performances more, or less, credible to audiences at
certain junctures. For example, many authenticity performances appear in
high frequencies for opposing and associated candidates in similar
years. In such, politicians adapt to perform authenticities audiences
``want to hear''. Finally, in both cases in recent years, debates became
the setting in which authenticities are performed most frequently
whereas interviews became the setting in which authenticities are
performed least frequently. Debates are large-scale media events that
produce ``sticky'' political bites charged with imagery that circulate
more than ever in democracies. Relatedly, social media platforms give
politicians diverse outlets to interact directly with audiences,
bypassing journalists in interviews.

Alexander (2011, 85) argues that the ``challenge for social performance
is to make its component parts invisible''. For social scientists, the
challenge has long been to understand when, why, and how political
discourses matter in democracies. Looking at politics as performances
emphasizes the performer's role, the script, the stage, and the
audience, rather than only focusing on the content (or an specific
interpretation of discursive content), and places agency with both
audiences and performers. Authenticity performances, as a framework,
offers a less contentious alternative to understand what discourses, as
anti-PC, are, how they change over time, and why they might matter for
political outcomes. Engaging just with meanings in political discourses
can misplace the logic of why electorates and politicians behave as they
do, while contributing to further polarization by passing on the blame
for ``undesirable'' political outcomes to a lumped together group of
``old, uneducated, or poor'' electorates. Even more worrisome, a
misplaced engagement with political discourses might reveal biased
answers to contemporary political issues moving from discourse to
policy, as in migration. As such, alternative approaches, as
authenticity performances, might be an useful extra tool in our Swiss
knife for engaging with political discourses.

\hypertarget{references}{%
\section{References}\label{references}}

\hypertarget{refs}{}
\begin{CSLReferences}{1}{0}
\leavevmode\vadjust pre{\hypertarget{ref-alexander2010}{}}%
Alexander, Jeffrey C. 2010. \emph{The Performance of Politics: Obama's
Victory and the Democratic Struggle for Power}. Oxford University Press.

\leavevmode\vadjust pre{\hypertarget{ref-alexander2011}{}}%
---------. 2011. \emph{Performance and Power}. Polity.

\leavevmode\vadjust pre{\hypertarget{ref-alexander2006}{}}%
Alexander, Jeffrey C, Bernhard Giesen, and Jason L Mast. 2006.
\emph{Social Performance: Symbolic Action, Cultural Pragmatics, and
Ritual}. Cambridge University Press.

\leavevmode\vadjust pre{\hypertarget{ref-allison2009}{}}%
Allison, Paul D. 2009. \emph{Fixed Effects Regression Models}. SAGE
publications.

\leavevmode\vadjust pre{\hypertarget{ref-aslanidis2016}{}}%
Aslanidis, Paris. 2016. {``Is Populism an Ideology? A Refutation and a
New Perspective.''} \emph{Political Studies} 64 (1\_suppl): 88--104.

\leavevmode\vadjust pre{\hypertarget{ref-baker2020}{}}%
Baker, Andy, Barry Ames, and Lúcio Rennó. 2020. \emph{Persuasive Peers:
Social Communication and Voting in Latin America}. Princeton University
Press.

\leavevmode\vadjust pre{\hypertarget{ref-beard1993}{}}%
Beard, Henry, and Christopher Cerf. 1993. \emph{The Official Politically
Correct Dictionary and Handbook}. Villard Books.

\leavevmode\vadjust pre{\hypertarget{ref-berman2011}{}}%
Berman, Paul. 2011. \emph{Debating PC: The Controversy over Political
Correctness on College Campuses}. Delta.

\leavevmode\vadjust pre{\hypertarget{ref-betz2001}{}}%
Betz, Hans-Georg. 2001. {``Exclusionary Populism in Austria, Italy, and
Switzerland.''} \emph{International Journal} 56 (3): 393--420.

\leavevmode\vadjust pre{\hypertarget{ref-blankenship1995}{}}%
Blankenship, Jane, and Deborah C Robson. 1995. {``A {`Feminine Style'}
in Women's Political Discourse: An Exploratory Essay.''}
\emph{Communication Quarterly} 43 (3): 353--66.

\leavevmode\vadjust pre{\hypertarget{ref-brinton2005}{}}%
Brinton, Laurel J, and Elizabeth Closs Traugott. 2005.
\emph{Lexicalization and Language Change}. Cambridge University Press.

\leavevmode\vadjust pre{\hypertarget{ref-brubaker2017}{}}%
Brubaker, Rogers. 2017. {``Why Populism?''} \emph{Theory and Society} 46
(5): 357--85.

\leavevmode\vadjust pre{\hypertarget{ref-brubaker2020}{}}%
---------. 2020. {``Populism and Nationalism.''} \emph{Nations and
Nationalism} 26 (1): 44--66.

\leavevmode\vadjust pre{\hypertarget{ref-bush1995}{}}%
Bush, Harold K. 1995. {``A Brief History of PC, with Annotated
Bibliography.''} \emph{American Studies International} 33 (1): 42--64.

\leavevmode\vadjust pre{\hypertarget{ref-bybee2015}{}}%
Bybee, Joan. 2015. \emph{Language Change}. Cambridge University Press.

\leavevmode\vadjust pre{\hypertarget{ref-carlo2018}{}}%
Carlo, Josnei, and Joao Kamradt. 2018. {``Bolsonaro e a Cultura Do
Politicamente Incorreto Na Politica Brasileira.''} \emph{Teoria e
Cultura} 13 (2).

\leavevmode\vadjust pre{\hypertarget{ref-cezar2021}{}}%
Cezar, Rodrigo Fagundes. 2020. {``Brazilian Presidential Speeches from
1985 to July 2020.''} Harvard Dataverse.
\url{https://doi.org/10.7910/DVN/M9UU09}.

\leavevmode\vadjust pre{\hypertarget{ref-chait2015}{}}%
Chait, Jonathan. 2015. {``Not a Very PC Thing to Say.''} \emph{New York
Magazine} 27.

\leavevmode\vadjust pre{\hypertarget{ref-christine2005}{}}%
Christine Banwart, Mary, and Mitchell S McKinney. 2005. {``A Gendered
Influence in Campaign Debates? Analysis of Mixed-Gender United States
Senate and Gubernatorial Debates.''} \emph{Communication Studies} 56
(4): 353--73.

\leavevmode\vadjust pre{\hypertarget{ref-coleman2000}{}}%
Coleman, Stephen. 2000. {``Televised Election Debates.''}
\emph{International Perspectives}.

\leavevmode\vadjust pre{\hypertarget{ref-conway2017}{}}%
Conway, Lucian Gideon III, Meredith A Repke, and Shannon C Houck. 2017.
{``Donald Trump as a Cultural Revolt Against Perceived Communication
Restriction: Priming Political Correctness Norms Causes More Trump
Support.''} \emph{Journal of Social and Political Psychology} 5 (1).

\leavevmode\vadjust pre{\hypertarget{ref-conway2009}{}}%
Conway, Lucian Gideon III, Amanda Salcido, Laura Janelle Gornick, Kate
Ashley Bongard, Meghan A Moran, and Chelsea Burfiend. 2009. {``When
Self-Censorship Norms Backfire: The Manufacturing of Positive
Communication and Its Ironic Consequences for the Perceptions of
Groups.''} \emph{Basic and Applied Social Psychology} 31 (4): 335--47.

\leavevmode\vadjust pre{\hypertarget{ref-d1991}{}}%
D'Souza, Dinesh. 1991. \emph{Illiberal Education: The Politics of Race
and Sex on Campus}. Simon; Schuster.

\leavevmode\vadjust pre{\hypertarget{ref-fairclough2003}{}}%
Fairclough, Norman. 2003. {``'Political Correctness': The Politics of
Culture and Language.''} \emph{Discourse \& Society} 14 (1): 17--28.

\leavevmode\vadjust pre{\hypertarget{ref-farias2001}{}}%
Farias, Edilsom Pereira de. 2001. {``Liberdade de Expressao e
Comunicacao.''}

\leavevmode\vadjust pre{\hypertarget{ref-feldstein1997}{}}%
Feldstein, Richard. 1997. \emph{Political Correctness: A Response from
the Cultural Left}. U of Minnesota Press.

\leavevmode\vadjust pre{\hypertarget{ref-fiorin2008}{}}%
Fiorin, Jose Luiz. 2008. {``A Linguagem Politicamente Correta.''}
\emph{Revista Linguasagem} 1 (1).

\leavevmode\vadjust pre{\hypertarget{ref-foley2012}{}}%
Foley, Megan. 2012. {``Sound Bites: Rethinking the Circulation of Speech
from Fragment to Fetish.''} \emph{Rhetoric and Public Affairs} 15 (4):
613--22.

\leavevmode\vadjust pre{\hypertarget{ref-fordahl2018}{}}%
Fordahl, Clayton. 2018. {``Authenticity: The Sociological Dimensions of
a Politically Consequential Concept.''} \emph{The American Sociologist}
49 (2): 299--311.

\leavevmode\vadjust pre{\hypertarget{ref-franceschet2016}{}}%
Franceschet, Susan, Jennifer M Piscopo, and Gwynn Thomas. 2016.
{``Supermadres, Maternal Legacies and Women's Political Participation in
Contemporary Latin America.''} \emph{Journal of Latin American Studies}
48 (1): 1--32.

\leavevmode\vadjust pre{\hypertarget{ref-freitas2013}{}}%
Freitas, Riva Sobrado de, and Matheus Felipe de Castro. 2013.
{``Liberdade de Expressao e Discurso Do Odio: Um Exame Sobre as
Possiveis Limitacoes a Liberdade de Expressao.''} \emph{Sequencia},
327--55.

\leavevmode\vadjust pre{\hypertarget{ref-goffman1956}{}}%
Goffman, Erving. 1956. {``The Presentation of Self in Everyday Life.
University of Edinburgh.''} \emph{Social Sciences Research Centre}.

\leavevmode\vadjust pre{\hypertarget{ref-goncalves2020}{}}%
Goncalves, Felipe, and Gabriela Goncalves. 2020. {``94.''} \emph{G1
Globo}.

\leavevmode\vadjust pre{\hypertarget{ref-gutmann1994}{}}%
Gutmann, Amy. 1994. {``Multiculturalism: Examining the Politics of
Recognition.''}

\leavevmode\vadjust pre{\hypertarget{ref-hall1994}{}}%
Hall, Stuart. 1994. {``Some "Politically Incorrect" Pathways Through
PC'in Sarah Dunant (Ed.) The War of the Words: The Political Correctness
Debate, 164-183.''} \emph{London: Virago}.

\leavevmode\vadjust pre{\hypertarget{ref-hawkins2009}{}}%
Hawkins, Kirk A. 2009. {``Is Chavez Populist? Measuring Populist
Discourse in Comparative Perspective.''} \emph{Comparative Political
Studies} 42 (8): 1040--67.

\leavevmode\vadjust pre{\hypertarget{ref-hawkins2018}{}}%
Hawkins, Kirk A, and Cristobal Rovira Kaltwasser. 2018. {``Measuring
Populist Discourse in the United States and Beyond.''} \emph{Nature
Human Behaviour} 2 (4): 241--42.

\leavevmode\vadjust pre{\hypertarget{ref-hughes2011}{}}%
Hughes, Geoffrey. 2011. \emph{Political Correctness: A History of
Semantics and Culture}. John Wiley; Sons.

\leavevmode\vadjust pre{\hypertarget{ref-laclau2005}{}}%
Laclau, Ernesto. 2005. \emph{On Populist Reason}. Verso.

\leavevmode\vadjust pre{\hypertarget{ref-loury1994}{}}%
Loury, Glenn C. 1994. {``Self-Censorship in Public Discourse: A Theory
of {`Political Correctness'} and Related Phenomena.''} \emph{Rationality
and Society} 6 (4): 428--61.

\leavevmode\vadjust pre{\hypertarget{ref-mckinney2013}{}}%
McKinney, Mitchell S, and Benjamin R Warner. 2013. {``Do Presidential
Debates Matter? Examining a Decade of Campaign Debate Effects.''}
\emph{Argumentation and Advocacy} 49 (4): 238--58.

\leavevmode\vadjust pre{\hypertarget{ref-mishra2017}{}}%
Mishra, Pankaj. 2017. \emph{Age of Anger: A History of the Present}.
Macmillan.

\leavevmode\vadjust pre{\hypertarget{ref-moffitt2016}{}}%
Moffitt, Benjamin. 2016. \emph{The Global Rise of Populism: Performance,
Political Style, and Representation}. Stanford University Press.

\leavevmode\vadjust pre{\hypertarget{ref-moffitt2014}{}}%
Moffitt, Benjamin, and Simon Tormey. 2014. {``Rethinking Populism:
Politics, Mediatisation and Political Style.''} \emph{Political Studies}
62 (2): 381--97.

\leavevmode\vadjust pre{\hypertarget{ref-montanaro2018}{}}%
Montanaro, Domenico. 2018. {``Warning to Democrats: Most Americans
Against US Getting More Politically Correct.''} NPR.

\leavevmode\vadjust pre{\hypertarget{ref-morato2017}{}}%
Morato, Edwiges, and Anna Christina Bentes. 2017. {``{`O Mundo Ta
Chato'}: Algumas Notas Sobre a Dimensao Sociocognitiva Do Politicamente
Correto Na Linguagem.''} \emph{Revista USP}, no. 115: 11--28.

\leavevmode\vadjust pre{\hypertarget{ref-mounk2018}{}}%
Mounk, Yascha. 2018. {``Americans Strongly Dislike PC Culture.''}
\emph{The Atlantic} 10.

\leavevmode\vadjust pre{\hypertarget{ref-mudde2004}{}}%
Mudde, Cas. 2004. {``The Populist Zeitgeist.''} \emph{Government and
Opposition} 39 (4): 541--63.

\leavevmode\vadjust pre{\hypertarget{ref-mudde2007}{}}%
---------. 2007. \emph{Populist Radical Right Parties in Europe}.
Cambridge: Cambridge university press.

\leavevmode\vadjust pre{\hypertarget{ref-norris2019}{}}%
Norris, Pippa, and Ronald Inglehart. 2019. \emph{Cultural Backlash:
Trump, Brexit, and Authoritarian Populism}. Cambridge University Press.

\leavevmode\vadjust pre{\hypertarget{ref-possenti1995}{}}%
Possenti, Sirio. 1995. {``A Linguagem Politicamente Correta e a Analise
Do Discurso.''} \emph{Revista de Estudos Da Linguagem} 3 (2): 123--40.

\leavevmode\vadjust pre{\hypertarget{ref-ragin1987}{}}%
Ragin, Charles C. 1987. \emph{The Comparative Method: Moving Beyond
Qualitative and Quantitative Strategies}. JSTOR.

\leavevmode\vadjust pre{\hypertarget{ref-rosenblum2020}{}}%
Rosenblum, Michael, Juliana Schroeder, and Francesca Gino. 2020. {``Tell
It Like It Is: When Politically Incorrect Language Promotes
Authenticity.''} \emph{Journal of Personality and Social Psychology} 119
(1): 75.

\leavevmode\vadjust pre{\hypertarget{ref-sposito2021}{}}%
Sposito, Henrique. 2021. \emph{Poldis: Tools for Analyzing Political
Discourse}. \url{https://github.com/henriquesposito/poldis}.

\leavevmode\vadjust pre{\hypertarget{ref-stiers2021}{}}%
Stiers, Dieter, Jac Larner, John Kenny, Sofia Breitenstein, Florence
Vallee-Dubois, and Michael Lewis-Beck. 2021. {``Candidate
Authenticity:'to Thine Own Self Be True'.''} \emph{Political Behavior}
43 (3): 1181--1204.

\leavevmode\vadjust pre{\hypertarget{ref-tamaki2020}{}}%
Tamaki, Eduardo Ryo, and Mario Fuks. 2020. {``Populism in Brazil's 2018
General Elections: An Analysis of Bolsonaro's Campaign Speeches.''}
\emph{Lua Nova: Revista de Cultura e Politica}, 103--27.

\leavevmode\vadjust pre{\hypertarget{ref-taylor1992}{}}%
Taylor, Charles. 1992. \emph{The Ethics of Authenticity}. Harvard
University Press.

\leavevmode\vadjust pre{\hypertarget{ref-valgarosson2021}{}}%
Valgarosson, Viktor Orri, Nick Clarke, Will Jennings, and Gerry Stoker.
2021. {``The Good Politician and Political Trust: An Authenticity Gap in
British Politics?''} \emph{Political Studies} 69 (4): 858--80.

\leavevmode\vadjust pre{\hypertarget{ref-dijk1997}{}}%
Van Dijk, Teun A et al. 1997. {``What Is Political Discourse
Analysis.''} \emph{Belgian Journal of Linguistics} 11 (1): 11--52.

\leavevmode\vadjust pre{\hypertarget{ref-weigel2016}{}}%
Weigel, Moira. 2016. {``Political Correctness: How the Right Invented a
Phantom Enemy.''} \emph{The Guardian} 30: 2016.

\leavevmode\vadjust pre{\hypertarget{ref-wc2014}{}}%
Weinmann, Amadeu de Oliveira, and Fabio Vacaro Culau. 2014. {``Notas
Sobre o Politicamente Correto.''} \emph{Estudos e Pesquisas Em
Psicologia} 14 (2): 628--45.

\leavevmode\vadjust pre{\hypertarget{ref-weyland2001}{}}%
Weyland, Kurt. 2001. {``Clarifying a Contested Concept: Populism in the
Study of Latin American Politics.''} \emph{Comparative Politics}, 1--22.

\leavevmode\vadjust pre{\hypertarget{ref-wood1994}{}}%
Wood, Julia T. 1994. {``Gendered Media: The Influence of Media on Views
of Gender.''} \emph{Gendered Lives: Communication, Gender, and Culture}
9: 231--44.

\end{CSLReferences}

\hypertarget{appendix}{%
\section{Appendix}\label{appendix}}

\begin{landscape}

\begingroup\fontsize{7}{9}\selectfont

\begin{longtabu} to \linewidth {>{\raggedright\arraybackslash}p{2cm}>{\raggedright\arraybackslash}p{6cm}>{\raggedright\arraybackslash}p{8cm}}
\caption{\label{tab:codebook}Authenticity Performances Codebook}\\
\toprule
Authenticity Performance & Lexicon English & Lexicon Portuguese\\
\midrule
\endfirsthead
\caption[]{Authenticity Performances Codebook \textit{(continued)}}\\
\toprule
Authenticity Performance & Lexicon English & Lexicon Portuguese\\
\midrule
\endhead

\endfoot
\bottomrule
\endlastfoot
\textbf{Truth Telling} & am telling the truth, are telling the truth, is telling the truth, the truth is, this is the truth, not lying, not lies, no lies, not telling you lies, is honest, am honest, is being honest, are being honest, are honest, honesty, is sincere, are sincere, am sincere, is being sincere, are being sincere, is true, are true, not a liar, bottom of my heart, I swear, I reassure, we reassure, I assure, we assure, be assured, is truthful, are truthful, am truthful, is being truthful, are being truthful, I know that, is evident, are evident, I am sure, trust me, am frank, are frank, is frank, being frank, is upfront, are upfront, am upfront, being upfront, will come clean, am coming clean, are coming clean, is straightforward, are straightforward, being straightforward, believe me, I am certain, no bullshit, not bullshitting & a verdade e, esta e a verdade, digo a verdade, dizemos a verdade, pura verdade, n<U+00E3>o e mentira, n<U+00E3>o estou mentindo, e honesto, sou honesto, somos honesto, sendo honesto, a honestidade, ser sincero, e sincero, com sinceridade, e verdade, s<U+00E3>o verdadeiras, n<U+00E3>o sou mentiroso, n<U+00E3>o minto, fundo do meu cora<U+00E7><U+00E3>o, sou verdadeiro, somos verdadeiros, tenho certeza, certeza absoluta, confia em mim, confie em mim, pode confiar, sou franco, somos francos, fraqueza, falando a verdade, falo a verdade, falamos a verdade, acredite em mim, pode acreditar, podem acreditar, eu tenho certeza, isso e a verdade, somos honestos, com honestidade, toda a sinceridade, com sinceridade, toda sinceridade, sou confi<U+00E1>vel, somos confi<U+00E1>veis, as coisas s<U+00E3>o assim, a realidade das coisas, juro por deus, com certeza, digo com precis<U+00E3>o, veracidade, premissa, afirmo para voc<U+00EA>s, isso e como aconteceu, falar umas verdades\\
\textbf{lying accusations} & not truth, not the truth, not true, aren<U+2019>t true, isn<U+2019>t true, being untruthful, is lying, are lying, is a liar, are liars, is dishonest, are  dishonest, being dishonest, is fake, are fake, being fake, is corrupt, are corrupt, full of lies, not sincere, not being sincere, isn<U+2019>t sincere, aren<U+2019>t sincere, not honest, not being honest, is cheating, is a cheater, are cheaters, are cheating, are tricking, is tricking, be deceived, is deceiving, are deceiving, are a hypocrite, is a hypocrite, are being a hypocrite, is being a hypocrite, is crooked, are crooked, is misleading, are misleading, has double-standards, are sneaky, is sneaky,  has two faces, two-faced, has double faces, double-faced, you are wrong, not correct, fooled by, do not believe, is misrepresenting, they misrepresent, is misrepresent, are misrepresent, pretends that, pretends to, is pretending, are pretending, keep pretending, breach your trust, breach of trust, is false, are false, being false, is misinforming, are misinforming, being misinformed, pretended, cut the crap, full of crap & n<U+00E3>o e verdade, n<U+00E3>o e verdadeiro, e mentiroso, est<U+00E1> mentindo, s<U+00E3>o mentiroso, e mentira, de mentira, tudo mentira, e desonesto, mentiram, mentiu, um desonesto, esse desonesto, de desonesto, s<U+00E3>o desonesto, e falso, s<U+00E3>o falsos, s<U+00E3>o corruptos, e corrupto, de corrupto, todos corrupto, n<U+00E3>o s<U+00E3>o sincero, n<U+00E3>o e sincero, n<U+00E3>o s<U+00E3>o honestos, n<U+00E3>o e honesto, s<U+00E3>o trapaceiros, e trapaceiro, eles trapaceiam, trapaceou, e enganar, ser enganado, v<U+00E3>o enganar, sendo enganados, e hip<U+00F3>crita, e enganador, e engana<U+00E7><U+00E3>o, duas caras, enganado por, n<U+00E3>o acredite, eles finge, ele finge, e fingimento, ela finge, quebrou a sua confian<U+00E7>a , quebra de confian<U+00E7>a,  e falso, s<U+00E3>o falsos, falsidade, e fic<U+00E7><U+00E3>o, hist<U+00F3>ria para boi dormir, historinha para boi dormir, e calunia, s<U+00E3>o calunias, difama<U+00E7><U+00E3>o, difamar, uma inverdade, s<U+00E3>o inverdades, e inverdade, isso e inven<U+00E7><U+00E3>o, essas s<U+00E3>o inven<U+00E7><U+00F5>es, isso e uma lenda, essas s<U+00E3>o ledas, tenta iludir, tentando iludir, uma farsa, tramoia, mal intencionado, mas inten<U+00E7><U+00F5>es, falta de informa<U+00E7><U+00E3>o, esta mal-informado, est<U+00E3>o mal-informados\\
\textbf{Consistency} & we delivered, I delivered, check and see, I keep my word, we keep our word, I kept my word, we kept our word, I keep my promise, I kept my promise, we keep our promise, as promised, we kept our promise, am responsible, I take responsibility, we take responsibility, we assume responsibility, we are accountable, we are responsible, our duty, my duty, give my word, giving my word, own up my, owning up my, accept responsibility, accept the blame, recognize my mistakes, admit I was wrong, I made mistakes, I guarantee, we guarantee, I can guarantee, we can guarantee, I promise, we promise, we can prove, I can prove, we proved, I proved, am reliable, rely on me, rely on us, be reassured, you can hold me accountable, you can hold us accountable, see with your own eyes, vote of confidence, our mission, my mission, my commitment, our commitment, during our government, during my government, while I was in charge & n<U+00F3>s entregamos, eu entreguei, veja com seus pr<U+00F3>prios olhos, cumpro minhas palavras, cumprimos nossas palavra, cumpri minha palavra, cumpro minhas promessas, nossas promessa, um compromisso, meu compromisso, tenho um compromisso com, eu sou respons<U+00E1>vel, eu assumo a responsabilidade, n<U+00F3>s somos respons<U+00E1>veis, n<U+00F3>s assumimos a responsabilidade, nosso dever, meu dever, dou minha palavra, fa<U+00E7>o uma promessa, fazer uma promessa, aceitar a responsabilidade, aceito a responsabilidade, aceitamos a responsabilidade, aceitar a culpa, meus erros, que errei, eu errei, eu garanto, eu posso garantir, eu prometo, podemos provar, posso provar, provaremos, eu provei, voto de confian<U+00E7>a, encarrego pessoalmente, encarreguei pessoalmente, estou comprometido, meu comprometimento, comprometimento com, o comprometimento, fazer o poss<U+00ED>vel, minha supervis<U+00E3>o, minha miss<U+00E3>o, nossa miss<U+00E3>o, no meu governo, no nosso governo, durante nosso governo, eu era encarregado, eu era o encarregado, fomos encarregados de\\
\textbf{Finger Pointing} & are inconsistent, is inconsistent, being inconsistent, are irresponsible, is irresponsible, being irresponsible, their fault, not my fault, not our fault, they left us with, they are responsible, are not responsible, aren<U+2019>t responsible, is not responsible, isn<U+2019>t responsible, costed us, false promises, lack accountability, lacking accountability, not kept their word, not kept his word, not kept her word, not kept promises, not kept the, not kept his, not kept her, not kept their, not keep their word, not keep his word, not keep her word, not keep the, didn<U+2019>t keep the, didn<U+2019>t keep her, didn<U+2019>t keep his, hasn<U+2019>t kept his, hasn<U+2019>t kept her, not recognize, he made mistakes, she made mistakes, they made mistakes, not our mistake, not my mistake, not take responsibility, not my responsibility, not accountable, him accountable, them accountable, her accountable, blame them, blame him, blame his, blame her, their blame, break promises, broken promises, has betrayed, they betrayed, betraying, will betray, has tricked, has lied, not deliver, didn<U+2019>t deliver, hasn<U+2019>t deliver, failed your obligations, failed in your obligations, failed his obligations, failed her obligations, failed in his duty, failed in her duty, failed his duty, failed her duty, failed your duties, stabbed in the back & e inconsistente, s<U+00E3>o inconsistente, e irrespons<U+00E1>vel, s<U+00E3>o irrespons<U+00E1>veis, culpa deles, a culpa n<U+00E3>o e minha, n<U+00E3>o e minha culpa, eles n<U+00F3>s deixaram, s<U+00E3>o respons<U+00E1>veis, e respons<U+00E1>vel, n<U+00F3>s custou, falsas promessas, falta de presta<U+00E7><U+00E3>o de contas, falharam, falhou, n<U+00E3>o cumpriu, n<U+00E3>o cumpriram, n<U+00E3>o reconheceu, n<U+00E3>o reconheceram, errou, erraram, n<U+00E3>o se responsabiliza, n<U+00E3>o me responsabilizo, culpa e sua, sua culpa, quebrar promessas, promessas quebradas, quebra de promessas, fala uma coisa e faz outra, fala uma coisa aqui e faz outra, falsas promessas, s<U+00E3>o trapaceiros, cometeu erros, cometeram erros, n<U+00E3>o reconhece, n<U+00E3>o reconheceu, assumiu a responsabilidade, promete uma coisa, promete o mundo, traiu a confian<U+00E7>a, traiu a sua confian<U+00E7>a, quebra de confian<U+00E7>a, quebraram sua confian<U+00E7>a, e falcatrua, foi falcatrua, cheio de falcatrua, houve fraude, houveram fraudes, fraudulento, uma negociata, facada nas costas, faltou com respeito, n<U+00E3>o faz o que promete, n<U+00E3>o fez o que promete, promessas em v<U+00E3>o, palavras em v<U+00E3>o, falta de comprometimento, falta de compromisso, houveram desvio, houve desvio, a culpa e do, cheio de promessas, a conta n<U+00E3>o fecha, n<U+00E3>o terminaram\\
\textbf{Origins} & I was born, I come from, we come from, I grew up, growing up in, my parents, my mom, my mother, my father, my dad, my family, raised me, I was raised, we were raised, we grew up, my background, being surrounded by, being exposed to, my siblings, going to school in, our local church, Sunday mass, Saturday mass, family tradition, tradition in my house, in our house, growing up, back in the day, my grandparents, in my town, in my state, in my region, our community, in my community, our town, our state, my hometown, our hometown, my home state, our home state, back home, our house, my house, our neighbourhood, in my district, I lived in, we lived in, we used to play, I used to play, I was thought & Eu nasci, Eu vim de, eu venho de, viemos de, cresci, n<U+00F3>s crescemos, meus pais, minha m<U+00E3>e, minha m<U+00E3>e, minha fam<U+00ED>lia, fui criado, fomos criados, minhas origens, meus irm<U+00E3>os, meu irm<U+00E3>o, minha irm<U+00E3>, tradi<U+00E7><U+00E3>o familiar, tradi<U+00E7><U+00E3>o em casa, crescendo, antigamente, meu av<U+00F4>, minha av<U+00F3>, meus av<U+00F3>s, na minha cidade, no meu estado, na minha regi<U+00E3>o, nossa comunidade, na minha comunidade, nossa cidade, nosso estado, cidade natal, estado de origem, minha casa, nossa casa, l<U+00E1> em casa, nosso bairro, no meu bairro, eu morava, viv<U+00ED>amos, na minha terra, de onde eu venho, missa de domingo, missa toda semana, brincava, eram outros tempos, fui educado, mor<U+00E1>vamos, eu morei, n<U+00F3>s moramos, de onde venho, eram tempos diferentes\\
\addlinespace
\textbf{Common Sense} & is common sense, are common sense, everyone knows, it is undeniable, stating the obvious, say the obvious, everyone agrees, we all know, common wisdom, the people know, popular knowledge, from experience, it is my experience, sound judgment, practical solution, practical choice, practical answer, pragmatic solution, pragmatic answer, pragmatic choice, realistic answer, let me tell you about, is obvious, are obvious, obvious answer, obvious solution, as we all learned, we have all learned that, do not need to tell you that, the reality is, there is no logic, it does not make sense, it doesn<U+2019>t make sense, we know it does not work, no one disagrees that, no person disagrees, there is not a person, there is not a human being, there is not a family, there is not an American, there is no single citizen, there is not one single person, there is not one single human being, there is not one single family, there is not one single American, there is not one single citizen, there is not one single person, there is not one human being, there is not one family, there is not one American & senso comum, bom senso, todos sabem, afirmando o <U+00F3>bvio, todos concordam, todos sabemos, sabemos todos, todos n<U+00F3>s sabemos, sabedoria popular, por experi<U+00EA>ncia, e minha experi<U+00EA>ncia, sou pr<U+00E1>tico, tem que ser pr<U+00E1>tico, devemos ser pr<U+00E1>tico, sendo pr<U+00E1>tico, sou pragm<U+00E1>tico, tem que ser pragm<U+00E1>tico, devemos ser  pragm<U+00E1>tico, sendo pragm<U+00E1>tico, sou realista, sendo realista, sejamos realista, realisticamente falando, e <U+00F3>bvio, como todos n<U+00F3>s aprendemos, como sabemos, n<U+00E3>o preciso te dizer, o povo sabe, agente aprendeu, n<U+00F3>s aprendemos, n<U+00F3>s sabemos, n<U+00E3>o tem logica, como aprendemos, n<U+00E3>o faz sentido, n<U+00E3>o fazem sentido, estamos cansados de saber, sabemos que n<U+00E3>o funciona, ningu<U+00E9>m discorda que, n<U+00E3>o tem uma pessoa, n<U+00E3>o existe uma pessoa, n<U+00E3>o h<U+00E1> uma pessoa, n<U+00E3>o existe um ser humano, n<U+00E3>o tem um ser humano, n<U+00E3>o h<U+00E1> um ser humano, n<U+00E3>o tem uma fam<U+00ED>lia, n<U+00E3>o existe uma fam<U+00ED>lia, n<U+00E3>o h<U+00E1> uma fam<U+00ED>lia, n<U+00E3>o tem um brasileiro, n<U+00E3>o h<U+00E1> um Brasileiro, n<U+00E3>o existe um brasileiro, n<U+00E3>o tem uma brasileira, n<U+00E3>o h<U+00E1> uma Brasileira, n<U+00E3>o existe uma brasileira\\
\textbf{Anti-PC} & politically correct, political correctness, PC, plain speaking, speaking my mind, speak my mind, say what I think, saying what I think, not going to pretend, not pretend, speak what you think, not what you want to hear, not butter up, not beat around the bush, cut to the chase, just being real, saying what everyone thinks, say what everyone is thinking, speaking plainly, coloured people, negro, retarded, nigger, third world, oriental people, crippled people, is crippled, culturally deprived, drug addict, junkie, drunk, fat people, fat person, fat population, handicapped, homosexual faggot, deviant, perverted, illegals, illegal immigrants, illegal alien, Jew, non-white, prostitutes, promiscuous, stupid, tribe, underdeveloped & politicamente correto, falar francamente, falando francamente, falar o que penso, falo o que penso, falando o que penso, dizer o que penso, papas na l<U+00ED>ngua, n<U+00E3>o vou fingir, n<U+00E3>o estou aqui para agradar, falar o que voc<U+00EA> pensa, o que voc<U+00EA> quer ouvir, n<U+00E3>o adulterar, n<U+00E3>o rodeio, n<U+00E3>o dou rodeio, direto ao ponto, dizer o que todos pensam, dizendo o que penso, dizendo o que todos pensam, dizer o que todos est<U+00E3>o pensando, n<U+00E3>o vou amaciar, n<U+00E3>o d<U+00E1> para amaciar, gordos, retardado, retardada, veado, popula<U+00E7><U+00E3>o preta, os pretos, as pretas, terceiro mundo, viciado em drogas, b<U+00EA>bado, drogado, sem cultura, pervertidos, prom<U+00ED>scuo, imbecil, estupido, aleijado, defeituoso, incapacitado, inv<U+00E1>lido, mongoloide, deficiente mental, defici<U+00EA>ncia mental, o incapacitado, a incapacitada, travesti, homossexualismo\\
\textbf{Territory} & have been to, have visited, came all the way to, back from, will visit, saw first-hand, see first-hand, we visited, I visited, we visited, travelled to, traveling to, spend a few days in, spent some time in, spent time in, met great people in, we were hosted, I was hosted, our time in, my time in, our visit, spent a lot of time in, were many times in, got to know the whole country, got to know all the states & estive em, visitei, voltou de, voltei de, voltando de, voltamos de, estive em, estivemos em, visitar<U+00E1>, visitarei, vi em primeira m<U+00E3>o, ver em primeira m<U+00E3>o, visitamos, viajei para, passei alguns dias em, passei algum tempo em, passei um tempo, conheci <U+00F3>timas pessoas, conhecemos <U+00F3>timas pessoas em, fomos hospedados, minha passagem, nossa passagem, nossa visita, fui muitas vezes para, estive muitas vezes em, passei muito tempo em, meu tempo em, estive por todo o Brasil, de norte a sul do pais, conhe<U+00E7>o todo o pais, conheci todo o pais, conheci todo o Brasil, conhe<U+00E7>o todo o Brasil\\*
\end{longtabu}
\endgroup{}

\end{landscape}

\begin{table}

\caption{\label{tab:table 4}Authenticity Performances and Election Years in Brazil and the US}
\centering
\begin{tabular}[t]{lc}
\toprule
  & Model 1\\
\midrule
Constant & \num{1.500}***\\
 & (\num{0.056})\\
Election Year & \num{-0.039}\\
 & (\num{0.085})\\
United States & \num{-0.302}***\\
 & (\num{0.073})\\
\midrule
Num.Obs. & \num{79}\\
R2 & \num{0.189}\\
R2 Adj. & \num{0.168}\\
AIC & \num{50.2}\\
BIC & \num{59.7}\\
F & \num{8.866}\\
RMSE & \num{0.32}\\
\bottomrule
\multicolumn{2}{l}{\rule{0pt}{1em}+ p $<$ 0.1, * p $<$ 0.05, ** p $<$ 0.01, *** p $<$ 0.001}\\
\end{tabular}
\end{table}

\begin{table}

\caption{\label{tab:table 5}Authenticity Performances by year in Brazil}
\centering
\resizebox{\linewidth}{!}{
\fontsize{7}{9}\selectfont
\begin{tabular}[t]{lc}
\toprule
  & Brazil\\
\midrule
\cellcolor{gray!6}{(Intercept)} & \cellcolor{gray!6}{\num{1.154}}\\
as.factor(date)1986 & \num{0.205}\\
\cellcolor{gray!6}{as.factor(date)1987} & \cellcolor{gray!6}{\num{0.261}}\\
as.factor(date)1988 & \num{0.301}\\
\cellcolor{gray!6}{as.factor(date)1989} & \cellcolor{gray!6}{\num{0.135}}\\
as.factor(date)1990 & \num{0.150}\\
\cellcolor{gray!6}{as.factor(date)1991} & \cellcolor{gray!6}{\num{-0.309}}\\
as.factor(date)1992 & \num{0.599}\\
\cellcolor{gray!6}{as.factor(date)1993} & \cellcolor{gray!6}{\num{-0.082}}\\
as.factor(date)1994 & \num{-0.201}\\
\cellcolor{gray!6}{as.factor(date)1995} & \cellcolor{gray!6}{\num{0.369}}\\
as.factor(date)1996 & \num{0.190}\\
\cellcolor{gray!6}{as.factor(date)1997} & \cellcolor{gray!6}{\num{-0.071}}\\
as.factor(date)1998 & \num{0.159}\\
\cellcolor{gray!6}{as.factor(date)1999} & \cellcolor{gray!6}{\num{0.271}}\\
as.factor(date)2000 & \num{0.338}\\
\cellcolor{gray!6}{as.factor(date)2001} & \cellcolor{gray!6}{\num{-0.055}}\\
as.factor(date)2002 & \num{0.167}\\
\cellcolor{gray!6}{as.factor(date)2003} & \cellcolor{gray!6}{\num{0.039}}\\
as.factor(date)2004 & \num{0.507}\\
\cellcolor{gray!6}{as.factor(date)2005} & \cellcolor{gray!6}{\num{0.512}}\\
as.factor(date)2006 & \num{0.231}\\
\cellcolor{gray!6}{as.factor(date)2007} & \cellcolor{gray!6}{\num{0.311}}\\
as.factor(date)2008 & \num{0.315}\\
\cellcolor{gray!6}{as.factor(date)2009} & \cellcolor{gray!6}{\num{0.023}}\\
as.factor(date)2010 & \num{0.227}\\
\cellcolor{gray!6}{as.factor(date)2011} & \cellcolor{gray!6}{\num{1.393}}\\
as.factor(date)2012 & \num{1.561}\\
\cellcolor{gray!6}{as.factor(date)2013} & \cellcolor{gray!6}{\num{0.704}}\\
as.factor(date)2014 & \num{1.220}\\
\cellcolor{gray!6}{as.factor(date)2015} & \cellcolor{gray!6}{\num{1.434}}\\
as.factor(date)2016 & \num{0.861}\\
\cellcolor{gray!6}{as.factor(date)2017} & \cellcolor{gray!6}{\num{0.165}}\\
as.factor(date)2018 & \num{-0.012}\\
\cellcolor{gray!6}{as.factor(date)2019} & \cellcolor{gray!6}{\num{0.136}}\\
as.factor(date)2020 & \num{0.220}\\
\cellcolor{gray!6}{as.factor(date)2021} & \cellcolor{gray!6}{\num{0.216}}\\
\midrule
Num.Obs. & \num{37}\\
\cellcolor{gray!6}{R2} & \cellcolor{gray!6}{\num{1.000}}\\
AIC & \num\\
\cellcolor{gray!6}{BIC} & \cellcolor{gray!6}{\num}\\
F & \\
\cellcolor{gray!6}{RMSE} & \cellcolor{gray!6}{\num{0.00}}\\
\bottomrule
\multicolumn{2}{l}{\rule{0pt}{1em}+ p $<$ 0.1, * p $<$ 0.05, ** p $<$ 0.01, *** p $<$ 0.001}\\
\end{tabular}}
\end{table}

\begin{table}

\caption{\label{tab:table 6}Authenticity Performances by Setting in Brazil and the US}
\centering
\begin{tabular}[t]{lcc}
\toprule
  & Brazil & US\\
\midrule
SettingCampaign Rally & \num{0.413}+ & \num{0.484}***\\
 & (\num{0.244}) & (\num{0.115})\\
SettingDebate & \num{0.686}* & \num{-0.054}\\
 & (\num{0.258}) & (\num{0.167})\\
SettingInterview & \num{-0.225} & \num{-0.224}*\\
 & (\num{0.139}) & (\num{0.098})\\
\midrule
Num.Obs. & \num{82} & \num{119}\\
R2 & \num{0.258} & \num{0.342}\\
R2 Adj. & \num{-0.430} & \num{-0.050}\\
AIC & \num{84.7} & \num{92.7}\\
BIC & \num{94.4} & \num{103.8}\\
RMSE & \num{0.39} & \num{0.35}\\
\bottomrule
\multicolumn{3}{l}{\rule{0pt}{1em}+ p $<$ 0.1, * p $<$ 0.05, ** p $<$ 0.01, *** p $<$ 0.001}\\
\end{tabular}
\end{table}

\end{document}
